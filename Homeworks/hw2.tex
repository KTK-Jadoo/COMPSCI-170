% Search for all the places that say "PUT SOMETHING HERE".

\documentclass[11pt]{article}
\usepackage{amsmath,textcomp,amssymb,geometry,graphicx,enumerate}

\def\Name{Kartikeya Sharma}  % Your name
\def\SID{3037376860}  % Your student ID number
\def\Homework{2 } % Number of Homework
\def\Session{Fall 2024}


\title{COMPSCI 170 Fall 2024 --- Homework \Homework Solutions}
\author{\Name, SID \SID}
\markboth{CS70--\Session\  Homework \Homework\ \Name}{CS170 -- \Session\ Homework \Homework\ \Name}
\pagestyle{myheadings}
\date{\today}

\newenvironment{qparts}{\begin{enumerate}[{(}a{)}]}{\end{enumerate}}
\def\endproofmark{$\Box$}
\newenvironment{proof}{\par{\bf Proof}:}{\endproofmark\smallskip}

\textheight=9in
\textwidth=6.5in
\topmargin=-.75in
\oddsidemargin=0.25in
\evensidemargin=0.25in


\begin{document}
\maketitle

\section*{1. Study Groups}

none



\newpage
\section*{2. Quantiles}
\textbf{1. Algorithm Description}

\begin{enumerate}
    \item \textbf{Initialize:}
        \begin{itemize}
            \item Define target ranks:
                \[
                \text{positions} = \left\{ \frac{n}{k},\ \frac{2n}{k},\ \dots,\ \frac{(k - 1)n}{k} \right\}.
                \]
            \item Call \(\text{Quantiles}(A,\ 1,\ n,\ \text{positions})\).
        \end{itemize}
    \item \textbf{Recursive Function} \(\text{Quantiles}(A,\ s,\ e,\ \text{positions})\):
        \begin{enumerate}
            \item If \(\text{positions}\) is empty, return.
            \item Select median rank \(m\) from \(\text{positions}\).
            \item Use \textbf{Select} to find element \(v\) of rank \(m\) in \(A[s\,:\,e]\).
            \item Partition \(A[s\,:\,e]\) around \(v\) into:
                \begin{itemize}
                    \item Left subarray \(L\): elements less than \(v\).
                    \item Right subarray \(R\): elements greater than \(v\).
                \end{itemize}
            \item Output \(v\) as the boundary for rank \(m\).
            \item Divide \(\text{positions}\) into:
                \begin{itemize}
                    \item \(\text{positions}_L = \{ r \in \text{positions} \mid r < m \}\).
                    \item \(\text{positions}_R = \{ r \in \text{positions} \mid r > m \}\).
                \end{itemize}
            \item Recursively call:
                \begin{itemize}
                    \item \(\text{Quantiles}(A,\ s,\ s + |L| - 1,\ \text{positions}_L)\).
                    \item \(\text{Quantiles}(A,\ e - |R| + 1,\ e,\ \text{positions}_R)\).
                \end{itemize}
        \end{enumerate}
\end{enumerate}

\textbf{2. Proof of Correctness}

We prove by induction on the number of positions:

\begin{itemize}
    \item \textbf{Base Case:} If \(\text{positions}\) is empty, the algorithm correctly returns without action.
    \item \textbf{Inductive Step:} Assume correctness for fewer positions. The algorithm selects the correct element \(v\) of rank \(m\) using \textbf{Select}, outputs it as a boundary, and partitions the array and positions for recursive calls. By induction, the recursive calls correctly find remaining boundaries.
\end{itemize}

Thus, the algorithm correctly computes all \(k - 1\) quantile boundaries.

\textbf{3. Runtime Analysis}

At each recursion level:

\begin{itemize}
    \item The \textbf{Select} operation and partitioning take \(O(n)\) time collectively.
    \item Since \(k\) is a power of 2, the number of positions is halved each time, leading to \(O(\log k)\) recursion depth.
\end{itemize}

Total time complexity is \(O(n \log k)\), as there are \(O(\log k)\) levels with \(O(n)\) work each.



\newpage
\section*{3. Median of Medians}

\textbf{(a) Worst-case runtime of QuickSelect(A, n/2):}

If the pivot is always either the smallest or largest element in the array, one of the partitions will have size $n - 1$, and the other will have size 0. Thus, each recursive call only reduces the array size by 1, leading to the recurrence relation:

\[
T(n) = T(n-1) + O(n)
\]

Solving this recurrence, we get:

\[
T(n) = O(n + (n-1) + (n-2) + \dots + 1) = O(n^2)
\]

Thus, the worst-case runtime of QuickSelect(A, n/2) is $O(n^2)$ when the pivot is poorly chosen in every recursive call. 

A sequence of "bad" pivot choices that causes this would be always selecting the smallest element (or the largest) as the pivot.


\textbf{(b) Proof that at least $\frac{3n}{10}$ elements are $\leq p$ and $\geq p$:}

We use the "Median of Medians" strategy to choose a pivot. 

1. Group the array into $\left\lfloor \frac{n}{5} \right\rfloor$ groups of 5 elements each.
2. Find the median of each group (since each group has 5 elements, this takes constant time, i.e., $O(1)$ per group).
3. Let $M$ be the array of medians. The size of $M$ is approximately $\frac{n}{5}$.
4. Find the median of $M$, which is the pivot $p$.

To show that at least $\frac{3n}{10}$ elements are less than or equal to $p$ and that at least $\frac{3n}{10}$ elements are greater than or equal to $p$:

\begin{itemize}
    \item Consider the groups in step 1. For each group, after sorting, the median is the third smallest element.
    \item At least half of the medians in $M$ are less than or equal to $p$ (since $p$ is the median of $M$). Hence, there are at least $\frac{1}{2} \cdot \frac{n}{5} = \frac{n}{10}$ groups whose median is less than or equal to $p$.
    \item In each such group, there are at least 3 elements less than or equal to $p$ (since the median is the third smallest element).
    \item Therefore, the number of elements less than or equal to $p$ is at least:$\frac{3n}{10}$
\end{itemize}




Similarly, there are at least $\frac{3n}{10}$ elements greater than or equal to $p$ because the same reasoning applies symmetrically to the groups where the median is greater than or equal to $p$.

Thus, at least $\frac{3n}{10}$ elements are less than or equal to $p$, and at least $\frac{3n}{10}$ elements are greater than or equal to $p$.

\newpage

\textbf{(c) Worst-case runtime of DeterministicSelect(A, k):}

\textbf{Algorithm Description:}

\begin{enumerate}
    \item Divide the array $A$ into groups of 5.
    \item Find the median of each group. This takes $O(1)$ per group, so $O(n)$ time for the entire array.
    \item Use DeterministicSelect recursively to find the median of the medians. The size of the new array of medians is $\left\lceil \frac{n}{5} \right\rceil$, so this step takes $T\left( \frac{n}{5} \right)$ time.
    \item Partition the original array around the pivot $p$ (the median of medians). This takes $O(n)$ time.
    \item Recur on the appropriate subarray. Since at least $\frac{3n}{10}$ elements are guaranteed to be discarded in each recursive step, the size of the subproblem is at most $\frac{7n}{10}$.
\end{enumerate}

\textbf{Recurrence Relation:}

The time complexity is governed by the following recurrence relation:

\[
T(n) \leq T\left( \frac{n}{5} \right) + T\left( \frac{7n}{10} \right) + O(n)
\]

We now solve this recurrence.

\textbf{Runtime Analysis:}

We will prove by induction that $T(n) \leq c n$ for some constant $c$.

\textbf{Base Case:} For small values of $n$, $T(n)$ is constant, so the inequality holds.

\textbf{Inductive Step:} Assume $T(n) \leq c n$ for all smaller values of $n$. Now, consider $T(n)$:

\[
T(n) \leq T\left( \frac{n}{5} \right) + T\left( \frac{7n}{10} \right) + O(n)
\]

Using the inductive hypothesis, we have:

\[
T(n) \leq c \left( \frac{n}{5} \right) + c \left( \frac{7n}{10} \right) + O(n)
\]

\[
T(n) \leq c \left( \frac{n}{5} + \frac{7n}{10} \right) + O(n)
\]

\[
T(n) \leq c \left( \frac{2n}{10} + \frac{7n}{10} \right) + O(n)
\]

\[
T(n) \leq c \left( \frac{9n}{10} \right) + O(n)
\]

\[
T(n) \leq 0.9c n + O(n)
\]

For sufficiently large $n$, the constant factor $O(n)$ can be absorbed into $0.9c n$, and we get:

\[
T(n) \leq c n
\]

Thus, the worst-case runtime of DeterministicSelect is $O(n)$.


\newpage
\section*{4. The Resistance Problem Solution}


\textbf{Problem Summary:}

We have \(n\) players, and \(s\) of them are spies. The goal is to identify all the spies using \(O(s \log(n/s))\) missions. Each mission is successful if no spy is sent; otherwise, the mission fails. After each mission, we only know whether it succeeded or failed.

\textbf{Strategy:}

We will use a divide-and-conquer approach, reducing the problem size by splitting the players into groups and analyzing the mission outcomes. The goal is to ensure that, after each round of missions, we can eliminate a significant fraction of non-spies or spies.

\begin{enumerate}
    \item \textbf{Initial Splitting:}
        \begin{itemize}
            \item Divide the \(n\) players into \(x\) equal-sized groups, with approximately \( \frac{n}{x} \) players per group.
            \item Send each group on a mission.
            \item If a mission succeeds, there are no spies in that group.
            \item If a mission fails, there is at least one spy in that group.
        \end{itemize}
    
    \item \textbf{Choosing \(x\):}
        \begin{itemize}
            \item We want to choose \(x\) such that at least half the groups contain no spies. This ensures that after sending all \(x\) groups on missions, we can reduce the problem size by at least half.
            \item On average, each group will contain approximately \( \frac{s}{x} \) spies, and at most \(s\) missions will fail (one for each spy).
            \item To ensure that at least half the groups have no spies, we set \(x \approx \frac{n}{s}\), which balances the number of groups and the number of spies.
        \end{itemize}
    
    \item \textbf{Recursive Reduction:}
        \begin{itemize}
            \item After identifying the groups with spies, recursively apply the same strategy to these smaller groups.
            \item In each round, the problem size decreases by at least half, and we repeat the process until we isolate all spies.
        \end{itemize}
\end{enumerate}

---

\textbf{Proof of Correctness:}

The strategy ensures that after each round of missions, we can eliminate at least half of the non-spy groups, reducing the number of players we need to consider by a factor of at least 2. Since we start with \(n\) players and reduce the problem size exponentially, we can identify all spies in \(O(s \log(n/s))\) missions.

---

\textbf{Runtime Analysis:}

Let \(T(n, s)\) represent the number of missions required to find all \(s\) spies among \(n\) players. In each round, we:

\begin{itemize}
    \item Split the players into \(x \approx \frac{n}{s}\) groups.
    \item Send each group on a mission, which takes \(O\left(\frac{n}{s}\right)\) missions.
    \item Recursively solve the problem for the remaining groups, reducing the problem size by half.
\end{itemize}

Thus, the recurrence relation is:

\[
T(n, s) = O\left(\frac{n}{s}\right) + T\left(\frac{n}{2}, s\right)
\]

Solving this recurrence gives:

\[
T(n, s) = O(s \log(n/s))
\]

This is the desired time complexity, meaning that the total number of missions required is \(O(s \log(n/s))\).

---

\textbf{Notes:} For \(n = 8\) players and \(s = 2\) spies:

\begin{enumerate}
    \item Split the 8 players into 4 groups of 2 players each.
    \item Send each group on a mission:
        \begin{itemize}
            \item If a mission succeeds, the group contains no spies.
            \item If a mission fails, the group contains at least one spy.
        \end{itemize}
    \item Suppose two missions fail. Now we know that the spies are in two of the groups.
    \item Recursively apply the same strategy to the two groups containing spies. After another round of missions, we will identify both spies.
\end{enumerate}

Thus, in this case, we can find all spies in 4 missions.


\newpage
\section*{5. Poker}
\textbf{(a) Algorithm to Determine if a Given Player \(x\) is a Truth-teller in \(O(n)\)}

\textbf{Algorithm Description:}
\begin{itemize}
    \item For each player \(y \neq x\), perform the following query:
        \begin{itemize}
            \item Ask player \(x\) if \(y\) tells the truth or bluffs.
            \item Ask player \(y\) if \(x\) tells the truth or bluffs.
        \end{itemize}
    \item If player \(x\) gives consistent answers across all queries, we conclude that \(x\) is a truth-teller.
    \item If any inconsistency is observed (i.e., different answers when queried about different players), then \(x\) is a bluffer.
\end{itemize}

\textbf{Proof of Correctness:}
\begin{itemize}
    \item Truth-tellers always tell the truth about others, so if \(x\) is a truth-teller, their answers will always be consistent.
    \item Bluffers may lie inconsistently about other players. Therefore, any inconsistency in \(x\)'s responses indicates that \(x\) is bluffing.
    \item Since every query checks the consistency of \(x\)'s answers, we can reliably determine whether \(x\) is a truth-teller or not.
\end{itemize}

\textbf{Runtime Analysis:}
\begin{itemize}
    \item For each of the other \(n - 1\) players, we perform two queries (one with \(x\) and one with \(y\)), resulting in \(O(n)\) total queries.
\end{itemize}

\newpage
\textbf{(b) Finding a Truth-teller in \(O(n \log n)\)}

\textbf{Algorithm Description:}
\begin{itemize}
    \item \textbf{Divide the players into two groups:} Split the \(n\) players into two groups, \(G_1\) and \(G_2\), each containing approximately \(n/2\) players.
    \item \textbf{Query between groups:} For each pair of players \(x \in G_1\) and \(y \in G_2\), perform the following query:
        \begin{itemize}
            \item Ask \(x\) if \(y\) tells the truth or bluffs.
            \item Ask \(y\) if \(x\) tells the truth or bluffs.
        \end{itemize}
    \item \textbf{Recurse on the group with a truth-teller:} At least one of the two groups must contain a truth-teller because there are more truth-tellers than bluffers. Discard the group that does not contain a truth-teller and recurse on the remaining group.
    \item \textbf{Repeat until a truth-teller is found:} Continue the process, splitting the remaining group into smaller subgroups and querying them until only one player remains, who must be a truth-teller.
\end{itemize}

\textbf{Proof of Correctness:}
\begin{itemize}
    \item There are more truth-tellers than bluffers. Therefore, in each recursive step, at least one group will always contain a truth-teller.
    \item By querying players from different groups, we can reliably determine which group contains a truth-teller and discard the other group. This process reduces the problem size by half with each iteration.
    \item Eventually, we are left with a small enough group that contains a truth-teller, and the process continues until a truth-teller is identified.
\end{itemize}

\textbf{Runtime Analysis:}
\begin{itemize}
    \item In each recursive step, we split the players into two groups and perform \(O(n)\) queries (one query for each pair of players between the groups).
    \item The problem size is halved in each step, so the depth of recursion is \(O(\log n)\).
    \item Therefore, the total number of queries is \(O(n \log n)\), as there are \(O(\log n)\) levels of recursion, and each level involves \(O(n)\) queries.
\end{itemize}

\newpage
\textbf{(c) Finding a Truth-teller in \(O(n)\)}

\textbf{Algorithm Description:}
\begin{itemize}
    \item \textbf{Pair the players:} Divide the \(n\) players into pairs of two players each.
    \item \textbf{Query within each pair:} For each pair of players \(x\) and \(y\):
        \begin{itemize}
            \item Ask \(x\) if \(y\) tells the truth or bluffs.
            \item Ask \(y\) if \(x\) tells the truth or bluffs.
        \end{itemize}
    \item \textbf{Eliminate pairs:} If both players in a pair accuse each other of bluffing (i.e., \(x\) says \(y\) bluffs and \(y\) says \(x\) bluffs), discard both players from further consideration.
    \item \textbf{Reduce the problem size:} After eliminating the inconsistent pairs, you are left with fewer players.
    \item \textbf{Repeat until one player remains:} Continue pairing the remaining players and querying them until only one player is left, who must be a truth-teller.
\end{itemize}

\textbf{Proof of Correctness:}
\begin{itemize}
    \item Each time two players accuse each other of bluffing, at least one of them must be a bluffer. By eliminating both players, we reduce the number of potential bluffers, ensuring that a truth-teller remains in the end.
    \item The process of elimination guarantees that if there are more truth-tellers than bluffers, at least one truth-teller will not be eliminated and will remain at the end.
\end{itemize}

\textbf{Runtime Analysis:}
\begin{itemize}
    \item In each round, we form \(n/2\) pairs and perform 2 queries for each pair, resulting in \(O(n)\) queries.
    \item The process reduces the number of players by approximately half in each round. Since there are \(O(\log n)\) rounds, the total number of queries is \(O(n)\).
\end{itemize}

\end{document}
