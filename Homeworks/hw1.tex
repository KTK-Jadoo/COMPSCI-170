% Search for all the places that say "PUT SOMETHING HERE".

\documentclass[11pt]{article}
\usepackage{amsmath,textcomp,amssymb,geometry,graphicx,enumerate}

\def\Name{Kartikeya Sharma}  % Your name
\def\SID{3037376860}  % Your student ID number
\def\Homework{1 } % Number of Homework
\def\Session{Fall 2024}


\title{COMPSCI 170 Fall 2024 --- Homework \Homework Solutions}
\author{\Name, SID \SID}
\markboth{CS70--\Session\  Homework \Homework\ \Name}{CS170 -- \Session\ Homework \Homework\ \Name}
\pagestyle{myheadings}
\date{\today}

\newenvironment{qparts}{\begin{enumerate}[{(}a{)}]}{\end{enumerate}}
\def\endproofmark{$\Box$}
\newenvironment{proof}{\par{\bf Proof}:}{\endproofmark\smallskip}

\textheight=9in
\textwidth=6.5in
\topmargin=-.75in
\oddsidemargin=0.25in
\evensidemargin=0.25in


\begin{document}
\maketitle

\section*{1. Study Groups}

none



\newpage
\section*{2. Course Policies}
\begin{qparts}
\item
No alternate exams are planned. \newline
Midterm 1: Wednesday, 10/2/2024, 7:00 PM - 9:00 PM \newline
Midterm 2: Tuesday, 11/5/2024, 7:00 PM - 9:00 PM \newline
Final: Tuesday, 12/17/2024, 8:00 AM - 11:00 AM

\item
Friday, 10:00 PM. The other 2 hours are mainly to cater to submission issues.

\item
Use the hw drops.

\item
Ed.

\item
“I have read and understood the course syllabus and policies.” - Kartikeya Sharma

\end{qparts}


\newpage
\section*{3. Understanding Academic Integrity}
\begin{qparts}
\item
Not OK.

\item
Not OK.

\item
Not OK.

\item
OK.

\end{qparts}


\newpage
\section*{4. Log Identities}
\begin{align*}
\text{(a)} & \quad \frac{\ln(x)}{\ln(y)} = \log_y(x) \\
\text{(b)} & \quad \ln(x) + \ln(y) = \ln(xy) \\
\text{(c)} & \quad \ln(x) - \ln(y) = \ln\left(\frac{x}{y}\right) \\
\text{(d)} & \quad 170 \cdot \ln(x) = \ln(x^{170})
\end{align*}

\newpage
\section*{5. Asymptotics Practice}

\begin{enumerate}

\item 
\( f(n) = n^2 + 5n \), \( g(n) = 1000(n+1)^2 \)
   \rightarrow \( f = O(g) \) \textbf{ and } \( g = O(f) \)

\item 
\( f(n) = 5n^3 \), \( g(n) = n^3 + (\log n)^{10} \)
   \rightarrow \( f = O(g) \) \textbf{ and } \( g = O(f) \)

\item 
\( f(n) = n^{100} \), \( g(n) = (1.01)^n \)
   \rightarrow \( f = O(g) \)

\item 
\( f(n) = (\log n)^{10} \), \( g(n) = n^{0.1} \)
   \rightarrow \( f = O(g) \)

\item 
\( f(n) = n \cdot 2^n \), \( g(n) = 3^n \)
   \rightarrow \( f = O(g) \)

\item 
\( f(n) = n! \), \( g(n) = n^n \)
   \rightarrow \( f = O(g) \)

\item 
\( f(n) = 1 + b + b^2 + \dots + b^n \), \( g(n) = b^n \) for arbitrary constant \( b > 0 \)
\begin{qparts}
\item If \( b > 1 \), \( f = O(g) \) \textbf{and} \( g = O(f) \).
\item If \( b = 1 \), \( g = O(f) \).
\item If \( 0 < b < 1 \), \( g = O(f) \).
\end{qparts}

\end{enumerate}



\newpage
\section*{6. Recurrence Relations Solutions}

\begin{qparts}
        
\item 
\( T(n) = 2T(n/3) + 5n \)

Using the expansion method.

Expanding the recurrence:

\[
T(n) = 2T\left(\frac{n}{3}\right) + 5n
\]

Substitute \( T\left(\frac{n}{3}\right) \) using the same recurrence:

\[
T\left(\frac{n}{3}\right) = 2T\left(\frac{n}{9}\right) + 5\left(\frac{n}{3}\right)
\]

Substituting back:

\[
T(n) = 2\left[2T\left(\frac{n}{9}\right) + 5\left(\frac{n}{3}\right)\right] + 5n = 4T\left(\frac{n}{9}\right) + \frac{10n}{3} + 5n
\]

Continue expanding:

\[
T(n) = 4\left[2T\left(\frac{n}{27}\right) + 5\left(\frac{n}{9}\right)\right] + \frac{10n}{3} + 5n = 8T\left(\frac{n}{27}\right) + \frac{20n}{9} + \frac{10n}{3} + 5n
\]

Generalizing this:

\[
T(n) = 2^k T\left(\frac{n}{3^k}\right) + n\sum_{i=0}^{k-1} \frac{5 \cdot 2^i}{3^i}
\]

For large \( k \), \( \frac{n}{3^k} \) approaches 1, so \( T(1) \) is a constant. The geometric series sum:

\[
\sum_{i=0}^{k-1} \frac{2^i}{3^i} = \sum_{i=0}^{k-1} \left(\frac{2}{3}\right)^i
\]

This is a convergent geometric series with sum:

\[
\sum_{i=0}^{\infty} \left(\frac{2}{3}\right)^i = \frac{1}{1-\frac{2}{3}} = 3
\]

Thus:

\[
T(n) = n \cdot 3 + O(n)
\]

So, the solution is \( T(n) = \Theta(n) \).

\item 
 An algorithm \( A \) takes \( \Theta(n^2) \) time to partition the input into 5 subproblems of size \( n/5 \), and then recursively runs itself on 3 of those subproblems.

The recurrence relation for this scenario is:

\[
T(n) = 3T\left(\frac{n}{5}\right) + \Theta(n^2)
\]

Using the expansion method:

\[
T(n) = 3\left[3T\left(\frac{n}{25}\right) + \left(\frac{n}{5}\right)^2\right] + n^2 = 3^2T\left(\frac{n}{25}\right) + 3n^2\left(\frac{1}{25}\right) + n^2
\]

Continuing the expansion:

\[
T(n) = 3^k T\left(\frac{n}{5^k}\right) + n^2 \sum_{i=0}^{k-1} \frac{3^i}{5^{2i}}
\]

As \( k \) increases, \( \frac{n}{5^k} \) approaches 1, so \( T(1) \) is a constant. The series sum is:

\[
\sum_{i=0}^{\infty} \left(\frac{3}{25}\right)^i
\]

This is also a geometric series with sum:

\[
\sum_{i=0}^{\infty} \left(\frac{3}{25}\right)^i = \frac{1}{1-\frac{3}{25}} = \frac{25}{22}
\]

Thus:

\[
T(n) = n^2 \cdot \frac{25}{22} + O(n^2)
\]

So, the solution is \( T(n) = \Theta(n^2) \).

\end{qparts}

\end{document}
